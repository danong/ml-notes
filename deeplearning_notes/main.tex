\documentclass[11pt]{report}

% --- Packages ---
\usepackage[usenames, dvipsnames]{color} % Cool colors
\usepackage{enumerate, amsmath, amsthm, amssymb, algorithm, algpseudocode, pifont, subfig, fullpage, csquotes, dashrule, tikz, bbm, booktabs, bm, verbatim, url}
\usepackage[framemethod=TikZ]{mdframed}
\usepackage[numbers]{natbib}
\usepackage{hyperref}
\usepackage{listings}
\usepackage{tex/pythonhighlight}

% --- Misc. ---
\hbadness=10000 % No "underfull hbox" messages.
\setlength{\parskip}{0.5em}

% --- Commands ---
% COMMANDS:
% - bigmid: Dynamically sized mid bar.
% - spacerule: add a centered dashed line with space above and below
% - \dbox{#1}: Adds a nicely formatted slightly grey box around #1
% - \begin{dproof} ... \end{dproof}: A nicely formatted proof. Use \qedhere to place qed
% - \ddef{#1}{#2}: Makes a definition (and counts defs). #1 goes inside parens at beginning, #2 is actual def.
% - \begin{dtable}{#1} ... \end{dtable}: Makes a minimalist table. #1 is the alignment, for example: {clrr} would be a 4 column, center left right right table.

% Dynamically sized mid bar.
\newcommand{\bigmid}{\mathrel{\Big|}}

% ---- Nice Color Palette and Notes ----
\definecolor{dblue}{RGB}{98, 140, 190}
\definecolor{dlblue}{RGB}{216, 235, 255}
\definecolor{dgreen}{RGB}{124, 155, 127}
\definecolor{dpink}{RGB}{207, 166, 208}
\definecolor{dyellow}{RGB}{255, 248, 199}
\definecolor{dgray}{RGB}{46, 49, 49}

% TODO
\newcommand{\todo}[1]{\textcolor{red}{TODO: #1}}
\newcommand{\dnote}[1]{\textcolor{dblue}{Dave: #1}}

% URL
\newcommand{\durl}[1]{\textcolor{dblue}{\underline{\url{#1}}}}

% Circled Numbers
\newcommand*\circled[1]{\tikz[baseline=(char.base)]{\node[shape=circle,draw,inner sep=0.7pt] (char) {\footnotesize{#1}};}}
% From: http://tex.stackexchange.com/questions/7032/good-way-to-make-textcircled-numbers

% Under set numbered subset of equation
\newcommand{\numeq}[3]{\underset{\textcolor{#2}{\circled{#1}}}{\textcolor{#2}{#3}}}

% ---- Abbreviations -----
\newcommand{\tc}[2]{\textcolor{#1}{#2}}
\newcommand{\ubr}[1]{\underbrace{#1}}
\newcommand{\uset}[2]{\underset{#1}{#2}}
\newcommand{\eps}{\varepsilon}

% Typical limit:
\newcommand{\nlim}{\underset{n \rightarrow \infty}{\lim}}
\newcommand{\nsum}{\sum_{i = 1}^n}
\newcommand{\nprod}{\prod_{i = 1}^n}

% Add an hrule with some space
\newcommand{\spacerule}{\begin{center}\hdashrule{2cm}{1pt}{1pt}\end{center}}

% Mathcal and Mathbb
\newcommand{\mc}[1]{\mathcal{#1}}
\newcommand{\indic}{\mathbbm{1}}
\newcommand{\bE}{\mathbb{E}}

\newcommand{\ra}{\rightarrow}
\newcommand{\la}{\leftarrow}

% ---- Figures, Boxes, Theorems, Etc. ----

% Basic Image
\newcommand{\img}[2]{
\begin{center}
\includegraphics[scale=#2]{#1}
\end{center}}

% Put a fancy box around things.
\newcommand{\dbox}[1]{
\begin{mdframed}[roundcorner=4pt, backgroundcolor=gray!5]
\vspace{1mm}
{#1}
\end{mdframed}
}

%  --- PROOFS ---

% Inner environment for Proofs
\newmdenv[
  topline=false,
  bottomline=false,
  rightline = false,
  leftmargin=10pt,
  rightmargin=0pt,
  innertopmargin=0pt,
  innerbottommargin=0pt
]{innerproof}

% Proof Command
\newenvironment{dproof}[1][Proof]{\begin{proof}[#1] \text{\vspace{2mm}} \begin{innerproof}}{\end{innerproof}\end{proof}\vspace{4mm}}

% Quick Definition
\newcounter{DaveDefCounter}
\setcounter{DaveDefCounter}{1}

\newcommand{\ddef}[2]
{
\begin{mdframed}[roundcorner=1pt, backgroundcolor=white]
\vspace{1mm}
{\bf Definition \theDaveDefCounter} (#1): {\it #2}
\stepcounter{DaveDefCounter}
\end{mdframed}
}

% Block Quote
\newenvironment{dblockquote}[2]{
\begin{blockquote}
#2
\vspace{-2mm}\hspace{10mm}{#1} \\
\end{blockquote}}

% Algorithm
\newenvironment{dalg}[1]
{\begin{algorithm}\caption{#1}\begin{algorithmic}}
{\end{algorithmic}\end{algorithm}}


% Quick Table
\newenvironment{dtable}[1]
{\begin{figure}[h]
\centering
\begin{tabular}{#1}\toprule}
{\bottomrule
\end{tabular}
\end{figure}}

% For numbering the last of an align*
\newcommand\numberthis{\addtocounter{equation}{1}\tag{\theequation}}

\DeclareMathOperator*{\argmin}{arg\,min}
\DeclareMathOperator*{\argmax}{arg\,max}

\newtheorem{conjecture}{Conjecture}[section]
\newtheorem{remark}{Remark}[section]
\newtheorem{theorem}{Theorem}[section]
\newtheorem{corollary}{Corollary}[theorem]
\newtheorem{lemma}[theorem]{Lemma}

% --- Meta Info ---
\title{deeplearning.ai Coursera Notes \thanks{\url{https://www.coursera.org/specializations/deep-learning}}}
\author{Daniel Ong\\ danielong1@gmail.com}
\date{March 2018}

% --- Begin Document ---
\begin{document}
\maketitle

\tableofcontents

\chapter*{About} 
\addcontentsline{toc}{chapter}{About}
This is my personal notes from all 5 courses of the \href{https://www.coursera.org/specializations/deep-learning}{Deep Learning specialization on Coursera by deeplearning.ai}. The primary motivations for creating this document are to create a useful reference for myself and to solidify my understanding of concepts and algorithms. Chapters and sections generally correspond to courses and weeks in the Coursera sequence but I split weeks that contain multiple topics into separate sections.

\begin{python}
def f(x):
    return x
\end{python}


\chapter{Neural Networks and Deep Learning}
\section*{Learning Objectives}
\begin{itemize}
  \item Understand the major technology trends driving Deep Learning
  \item Be able to build, train, and apply fully connected deep neural networks
  \item Know how to implement efficient (vectorized) neural networks
  \item Understand the key parameters in a neural network's architecture
\end{itemize}

\section{Introduction to Deep Learning}
Andrew Ng believes that the impact of artificial intelligence (AI) on society will be on the same magnitude as the impact of electricity on society a hundred years ago. Deep learning (DL) is the field of AI that has been improving rapidly and driving recent progress in AI research. DL is already a powerful and proven method for solving certain types of problems (e.g. advertising, image classification, etc). 

\img{neuron.png}{A biological neuron}
Artificial neural networks are loosely based on biological neural networks. In brains, neurons are connected to each other to form a neural network. Each neuron has several dendrites (input connections), a soma (sums input signals), and axon terminals (output connections). If the sum of the input signals surpasses a threshold, the neuron outputs an electrical signal down its axon to other neuron's dendrites.\footnote{This is an extremely simplified model.} This is the only time biological neurons will be discussed. All future references to neurons and neural networks refer to artificial neurons and artificial neural networks.

Similarly, artificial neurons receive inputs, applies a weight to the inputs, calculates the sum of their inputs, and passes an output to other neurons.
\ddef{Artificial Neuron}{An artificial neuron receives one or more inputs, x. Applies a weight to the input, W. }

To implement more complicated nonlinear functions with many input variables, multiple neurons are needed.\footnote{You can also create binomial combinations of variables to generate nonlinear functions but this does not scale well.} An artificial neural network is a collection of artificial neurons that are linked together. 

\ddef{Neural Network}{A neural network is a collection of artificial neurons that are linked together.}
\ddef{Shallow Neural Network}{A neural network with one hidden layer is a shallow neural network.}

\todo{Use proper notation in shallow NN diagram}

\def\layersep{2.5cm}
\begin{tikzpicture}[shorten >=1pt,->,draw=black!50, node distance=\layersep]
    \tikzstyle{every pin edge}=[<-,shorten <=1pt]
    \tikzstyle{neuron}=[circle,fill=black!25,minimum size=17pt,inner sep=0pt]
    \tikzstyle{input neuron}=[neuron, fill=green!50];
    \tikzstyle{output neuron}=[neuron, fill=red!50];
    \tikzstyle{hidden neuron}=[neuron, fill=blue!50];
    \tikzstyle{annot} = [text width=4em, text centered]

    % Draw the input layer nodes
    \foreach \name / \y in {1,...,4}
    % This is the same as writing \foreach \name / \y in {1/1,2/2,3/3,4/4}
        \node[input neuron, pin=left:Input \#\y] (I-\name) at (0,-\y) {};

    % Draw the hidden layer nodes
    \foreach \name / \y in {1,...,5}
        \path[yshift=0.5cm]
            node[hidden neuron] (H-\name) at (\layersep,-\y cm) {};

    % Draw the output layer node
    \node[output neuron,pin={[pin edge={->}]right:Output}, right of=H-3] (O) {};

    % Connect every node in the input layer with every node in the
    % hidden layer.
    \foreach \source in {1,...,4}
        \foreach \dest in {1,...,5}
            \path (I-\source) edge (H-\dest);

    % Connect every node in the hidden layer with the output layer
    \foreach \source in {1,...,5}
        \path (H-\source) edge (O);

    % Annotate the layers
    \node[annot,above of=H-1, node distance=1cm] (hl) {Hidden layer};
    \node[annot,left of=hl] {Input layer};
    \node[annot,right of=hl] {Output layer};
\end{tikzpicture}
\ddef{Deep Neural Network}{A neural network with multiple hidden layers is a deep neural network.}
\todo{Insert deep NN diagram}
\ddef{Deep Learning}{Deep Learning refers to the training of large neural networks.}
Neural networks were widely used in the 80s and early 90s but their popularity diminished in the late 90s. However, recent improvements in hardware and software have made neural networks state of the art again. Factors for deep learning's the recent popularity and success include:
\begin{itemize}
	\item Large quantity of data: we generate and save more data than ever before. More data generally means better NN performance.
	\item More compute power: In addition to general \href{https://en.wikipedia.org/wiki/Moore%27s_law}{Moore's Law} progress, we also have access to GPUs and distributed/cloud computing that speed up training. More compute power allows us to train bigger NNs. 
	\item Improved algorithms: Researchers continue to discover new NN architectures and algorithms that improve performance.
\end{itemize}
\section{Neural Network Basics}

\todo{Use proper formatting for superscripts/subscripts}
\dbox{
NOTATION

\begin{itemize}
	\item M is the number of training vectors
	\item Nx is the size of the input vector
	\item Ny is the size of the output vector
	\item X(1) is the first input vector
	\item Y(1) is the first output vector
	\item X = [x(1) x(2).. x(M)]
	\item Y = (y(1) y(2).. y(M))
\end{itemize}
}

\subsection{Logistic regression}
A single neuron can implement logistic regression. Let's take a look at linear regression from a machine learning point of view.
\ddef{Logistic regression}{Regression model where the dependent variable (DV) is categorical. Examples: classify a tumor as either benign or malignant based on features, classify an image as either containing a hotdog or not containing a hotdog.}
Our implementation of linear regression will have 3 components. The linear hypothesis function outputs our line of best fit. The cost function measures how accurate our hypothesis is. We use the gradient descent algorithm to minimize the cost function. 

\todo{Replace with proper notation from new course}

\dbox{
Hypothesis function: A function of x for fixed parameters. Given parameters $\theta$, function $h$ outputs a $y$ (output) value for any given $x$ (input) value.
$$h_{\theta}(x)=\theta_{0}+\theta_{1}x$$
}


\dbox{
Cost function or Squared error function: a way to measure the accuracy of our hypothesis function. A function of the parameters. It is the sum of the squares of the differences between our predicted output values and the given target variables. We want to choose parameters such that our hypothesis makes reasonably accurate predictions for our training examples. The cost function is a function of the parameters.

$$J(\theta_{0},\theta_{1}) = \frac{1}{2m}\sum_{i = 1}^{m}(h_\theta (y^{(i)}) - (x^{(i)}))^2$$
}


\section{Shallow Neural Network}

\section{Neural Network Basics}





\section{Deep Neural Network}


\chapter{Improving Deep Neural Networks}
\section{Practical Aspects of Deep Learning}
\section{Optimization Algorithms}
\section{Hyperparameter Tuning}
\section{Batch Normalization}
\section{Programming Frameworks}

\chapter{Structuring Machine Learning Projects}

\chapter{Convolutional Neural Networks}

\chapter{Sequence Models}


% --- Bibliography ---
\bibliographystyle{plainnat}
\bibliography{research}

\end{document}