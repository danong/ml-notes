\chapter{Neural Networks and Deep Learning}
\section*{Learning Objectives}
\begin{itemize}
  \item Understand fundamentals of neural networks
  \item Build a deep neural network that recognizes cats
\end{itemize}

\section{Introduction to Deep Learning}
Andrew Ng believes that the impact of artificial intelligence (AI) on society will be on the same magnitude as the impact of electricity on society a hundred years ago. Deep learning (DL) is the field of AI that has been improving rapidly and driving recent progress. DL is already a powerful and proven method for solving certain types of problems (e.g. advertising, image classification, etc). 


\section{Neural Network Basics}
Artificial neural networks are loosely based on biological neural networks. In brains, neurons are connected to each other to form a neural network. Each neuron has several dendrites (input connections), a soma (sums input signals), and axon terminals (output connections). If the sum of the input signals surpasses a threshold, the neuron outputs an electrical signal down its axon to other neuron's dendrites. 


Similarly, artificial neurons receive inputs, sum them, and passes an output to other neurons.  
\ddef{Artificial Neuron}{An artificial neuron receives one or more inputs, x. Applies a weight to the input, W. }

\ddef{Neural Network}{A neural network is a collection of nodes that apply an activation function onto input...}
\section{Shallow Neural Network}
\ddef{Shallow Neural Network}{A neural network is a collection of nodes that apply an activation function onto input...}
\section{Deep Neural Network}
\ddef{Deep Neural Network}{A neural network is a collection of nodes that apply an activation function onto input...}
\ddef{Deep Learning}{Deep Learning refers to the training of large neural networks.}